\section{Discussion}
A stacked autoencoder enjoys all the benefits of any deep network of greater expressive power.~Further, it often captures an useful ``hierarchical combination'' or ``part-whole decomposition'' of the input.~To view this, recall that an autoencoder endeavors to learn features that form an appropriate representation of its input.~The first layer of a stacked autoencoder tries to learn first-order features in the raw input (such as edges in an image).~The second layer of a stacked autoencoder tries to learn second-order features corresponding to patterns in the appearance of first-order features (e.g. in terms of what edges tend to occur together---for instance, to form contour or corner detectors).~Higher layers of the stacked autoencoder attempt to learn even higher-order features.