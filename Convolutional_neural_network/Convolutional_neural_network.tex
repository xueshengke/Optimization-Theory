
%% bare_jrnl.tex
%% V1.4b
%% 2015/08/26
%% by Michael Shell
%% see http://www.michaelshell.org/
%% for current contact information.
%%
%% This is a skeleton file demonstrating the use of IEEEtran.cls
%% (requires IEEEtran.cls version 1.8b or later) with an IEEE
%% journal paper.
%%
%% Support sites:
%% http://www.michaelshell.org/tex/ieeetran/
%% http://www.ctan.org/pkg/ieeetran
%% and
%% http://www.ieee.org/

%%*************************************************************************
%% Legal Notice:
%% This code is offered as-is without any warranty either expressed or
%% implied; without even the implied warranty of MERCHANTABILITY or
%% FITNESS FOR A PARTICULAR PURPOSE!
%% User assumes all risk.
%% In no event shall the IEEE or any contributor to this code be liable for
%% any damages or losses, including, but not limited to, incidental,
%% consequential, or any other damages, resulting from the use or misuse
%% of any information contained here.
%%
%% All comments are the opinions of their respective authors and are not
%% necessarily endorsed by the IEEE.
%%
%% This work is distributed under the LaTeX Project Public License (LPPL)
%% ( http://www.latex-project.org/ ) version 1.3, and may be freely used,
%% distributed and modified. A copy of the LPPL, version 1.3, is included
%% in the base LaTeX documentation of all distributions of LaTeX released
%% 2003/12/01 or later.
%% Retain all contribution notices and credits.
%% ** Modified files should be clearly indicated as such, including  **
%% ** renaming them and changing author support contact information. **
%%*************************************************************************


% *** Authors should verify (and, if needed, correct) their LaTeX system  ***
% *** with the testflow diagnostic prior to trusting their LaTeX platform ***
% *** with production work. The IEEE's font choices and paper sizes can   ***
% *** trigger bugs that do not appear when using other class files.       *** 
% The testflow support page is at:
% http://www.michaelshell.org/tex/testflow/



\documentclass[draftcls, onecolumn, 11pt]{IEEEtran}
%\documentclass[journal]{IEEEtran}
%
% If IEEEtran.cls has not been installed into the LaTeX system files,
% manually specify the path to it like:
% \documentclass[journal]{../sty/IEEEtran}

\usepackage{graphicx}
\usepackage{cite}
\usepackage{bm}
\usepackage{amsmath, amsfonts, amssymb}
\usepackage[colorlinks]{hyperref}

% Some very useful LaTeX packages include:
% (uncomment the ones you want to load)


% *** MISC UTILITY PACKAGES ***
%
%\usepackage{ifpdf}
% Heiko Oberdiek's ifpdf.sty is very useful if you need conditional
% compilation based on whether the output is pdf or dvi.
% usage:
% \ifpdf
%   % pdf code
% \else
%   % dvi code
% \fi
% The latest version of ifpdf.sty can be obtained from:
% http://www.ctan.org/pkg/ifpdf
% Also, note that IEEEtran.cls V1.7 and later provides a builtin
% \ifCLASSINFOpdf conditional that works the same way.
% When switching from latex to pdflatex and vice-versa, the compiler may
% have to be run twice to clear warning/error messages.



% *** CITATION PACKAGES ***
%
%\usepackage{cite}
% cite.sty was written by Donald Arseneau
% V1.6 and later of IEEEtran pre-defines the format of the cite.sty package
% \cite{} output to follow that of the IEEE. Loading the cite package will
% result in citation numbers being automatically sorted and properly
% "compressed/ranged". e.g., [1], [9], [2], [7], [5], [6] without using
% cite.sty will become [1], [2], [5]--[7], [9] using cite.sty. cite.sty's
% \cite will automatically add leading space, if needed. Use cite.sty's
% noadjust option (cite.sty V3.8 and later) if you want to turn this off
% such as if a citation ever needs to be enclosed in parenthesis.
% cite.sty is already installed on most LaTeX systems. Be sure and use
% version 5.0 (2009-03-20) and later if using hyperref.sty.
% The latest version can be obtained at:
% http://www.ctan.org/pkg/cite
% The documentation is contained in the cite.sty file itself.



% *** GRAPHICS RELATED PACKAGES ***
%
\ifCLASSINFOpdf
% \usepackage[pdftex]{graphicx}
% declare the path(s) where your graphic files are
% \graphicspath{{../pdf/}{../jpeg/}}
% and their extensions so you won't have to specify these with
% every instance of \includegraphics
% \DeclareGraphicsExtensions{.pdf,.jpeg,.png}
\else
% or other class option (dvipsone, dvipdf, if not using dvips). graphicx
% will default to the driver specified in the system graphics.cfg if no
% driver is specified.
% \usepackage[dvips]{graphicx}
% declare the path(s) where your graphic files are
% \graphicspath{{../eps/}}
% and their extensions so you won't have to specify these with
% every instance of \includegraphics
% \DeclareGraphicsExtensions{.eps}
\fi
% graphicx was written by David Carlisle and Sebastian Rahtz. It is
% required if you want graphics, photos, etc. graphicx.sty is already
% installed on most LaTeX systems. The latest version and documentation
% can be obtained at:
% http://www.ctan.org/pkg/graphicx
% Another good source of documentation is "Using Imported Graphics in
% LaTeX2e" by Keith Reckdahl which can be found at:
% http://www.ctan.org/pkg/epslatex
%
% latex, and pdflatex in dvi mode, support graphics in encapsulated
% postscript (.eps) format. pdflatex in pdf mode supports graphics
% in .pdf, .jpeg, .png and .mps (metapost) formats. Users should ensure
% that all non-photo figures use a vector format (.eps, .pdf, .mps) and
% not a bitmapped formats (.jpeg, .png). The IEEE frowns on bitmapped formats
% which can result in "jaggedy"/blurry rendering of lines and letters as
% well as large increases in file sizes.
%
% You can find documentation about the pdfTeX application at:
% http://www.tug.org/applications/pdftex



% *** MATH PACKAGES ***
%
%\usepackage{amsmath}
% A popular package from the American Mathematical Society that provides
% many useful and powerful commands for dealing with mathematics.
%
% Note that the amsmath package sets \interdisplaylinepenalty to 10000
% thus preventing page breaks from occurring within multiline equations. Use:
%\interdisplaylinepenalty=2500
% after loading amsmath to restore such page breaks as IEEEtran.cls normally
% does. amsmath.sty is already installed on most LaTeX systems. The latest
% version and documentation can be obtained at:
% http://www.ctan.org/pkg/amsmath



% *** SPECIALIZED LIST PACKAGES ***
%
%\usepackage{algorithmic}
% algorithmic.sty was written by Peter Williams and Rogerio Brito.
% This package provides an algorithmic environment fo describing algorithms.
% You can use the algorithmic environment in-text or within a figure
% environment to provide for a floating algorithm. Do NOT use the algorithm
% floating environment provided by algorithm.sty (by the same authors) or
% algorithm2e.sty (by Christophe Fiorio) as the IEEE does not use dedicated
% algorithm float types and packages that provide these will not provide
% correct IEEE style captions. The latest version and documentation of
% algorithmic.sty can be obtained at:
% http://www.ctan.org/pkg/algorithms
% Also of interest may be the (relatively newer and more customizable)
% algorithmicx.sty package by Szasz Janos:
% http://www.ctan.org/pkg/algorithmicx


% *** ALIGNMENT PACKAGES ***
%
%\usepackage{array}
% Frank Mittelbach's and David Carlisle's array.sty patches and improves
% the standard LaTeX2e array and tabular environments to provide better
% appearance and additional user controls. As the default LaTeX2e table
% generation code is lacking to the point of almost being broken with
% respect to the quality of the end results, all users are strongly
% advised to use an enhanced (at the very least that provided by array.sty)
% set of table tools. array.sty is already installed on most systems. The
% latest version and documentation can be obtained at:
% http://www.ctan.org/pkg/array


% IEEEtran contains the IEEEeqnarray family of commands that can be used to
% generate multiline equations as well as matrices, tables, etc., of high
% quality.


% *** SUBFIGURE PACKAGES ***
%\ifCLASSOPTIONcompsoc
%  \usepackage[caption=false,font=normalsize,labelfont=sf,textfont=sf]{subfig}
%\else
%  \usepackage[caption=false,font=footnotesize]{subfig}
%\fi
% subfig.sty, written by Steven Douglas Cochran, is the modern replacement
% for subfigure.sty, the latter of which is no longer maintained and is
% incompatible with some LaTeX packages including fixltx2e. However,
% subfig.sty requires and automatically loads Axel Sommerfeldt's caption.sty
% which will override IEEEtran.cls' handling of captions and this will result
% in non-IEEE style figure/table captions. To prevent this problem, be sure
% and invoke subfig.sty's "caption=false" package option (available since
% subfig.sty version 1.3, 2005/06/28) as this is will preserve IEEEtran.cls
% handling of captions.
% Note that the Computer Society format requires a larger sans serif font
% than the serif footnote size font used in traditional IEEE formatting
% and thus the need to invoke different subfig.sty package options depending
% on whether compsoc mode has been enabled.
%
% The latest version and documentation of subfig.sty can be obtained at:
% http://www.ctan.org/pkg/subfig


% *** FLOAT PACKAGES ***
%
%\usepackage{fixltx2e}
% fixltx2e, the successor to the earlier fix2col.sty, was written by
% Frank Mittelbach and David Carlisle. This package corrects a few problems
% in the LaTeX2e kernel, the most notable of which is that in current
% LaTeX2e releases, the ordering of single and double column floats is not
% guaranteed to be preserved. Thus, an unpatched LaTeX2e can allow a
% single column figure to be placed prior to an earlier double column
% figure.
% Be aware that LaTeX2e kernels dated 2015 and later have fixltx2e.sty's
% corrections already built into the system in which case a warning will
% be issued if an attempt is made to load fixltx2e.sty as it is no longer
% needed.
% The latest version and documentation can be found at:
% http://www.ctan.org/pkg/fixltx2e


%\usepackage{stfloats}
% stfloats.sty was written by Sigitas Tolusis. This package gives LaTeX2e
% the ability to do double column floats at the bottom of the page as well
% as the top. (e.g., "\begin{figure*}[!b]" is not normally possible in
% LaTeX2e). It also provides a command:
%\fnbelowfloat
% to enable the placement of footnotes below bottom floats (the standard
% LaTeX2e kernel puts them above bottom floats). This is an invasive package
% which rewrites many portions of the LaTeX2e float routines. It may not work
% with other packages that modify the LaTeX2e float routines. The latest
% version and documentation can be obtained at:
% http://www.ctan.org/pkg/stfloats
% Do not use the stfloats baselinefloat ability as the IEEE does not allow
% \baselineskip to stretch. Authors submitting work to the IEEE should note
% that the IEEE rarely uses double column equations and that authors should try
% to avoid such use. Do not be tempted to use the cuted.sty or midfloat.sty
% packages (also by Sigitas Tolusis) as the IEEE does not format its papers in
% such ways.
% Do not attempt to use stfloats with fixltx2e as they are incompatible.
% Instead, use Morten Hogholm'a dblfloatfix which combines the features
% of both fixltx2e and stfloats:
%
% \usepackage{dblfloatfix}
% The latest version can be found at:
% http://www.ctan.org/pkg/dblfloatfix




%\ifCLASSOPTIONcaptionsoff
%  \usepackage[nomarkers]{endfloat}
% \let\MYoriglatexcaption\caption
% \renewcommand{\caption}[2][\relax]{\MYoriglatexcaption[#2]{#2}}
%\fi
% endfloat.sty was written by James Darrell McCauley, Jeff Goldberg and
% Axel Sommerfeldt. This package may be useful when used in conjunction with
% IEEEtran.cls'  captionsoff option. Some IEEE journals/societies require that
% submissions have lists of figures/tables at the end of the paper and that
% figures/tables without any captions are placed on a page by themselves at
% the end of the document. If needed, the draftcls IEEEtran class option or
% \CLASSINPUTbaselinestretch interface can be used to increase the line
% spacing as well. Be sure and use the nomarkers option of endfloat to
% prevent endfloat from "marking" where the figures would have been placed
% in the text. The two hack lines of code above are a slight modification of
% that suggested by in the endfloat docs (section 8.4.1) to ensure that
% the full captions always appear in the list of figures/tables - even if
% the user used the short optional argument of \caption[]{}.
% IEEE papers do not typically make use of \caption[]'s optional argument,
% so this should not be an issue. A similar trick can be used to disable
% captions of packages such as subfig.sty that lack options to turn off
% the subcaptions:
% For subfig.sty:
% \let\MYorigsubfloat\subfloat
% \renewcommand{\subfloat}[2][\relax]{\MYorigsubfloat[]{#2}}
% However, the above trick will not work if both optional arguments of
% the \subfloat command are used. Furthermore, there needs to be a
% description of each subfigure *somewhere* and endfloat does not add
% subfigure captions to its list of figures. Thus, the best approach is to
% avoid the use of subfigure captions (many IEEE journals avoid them anyway)
% and instead reference/explain all the subfigures within the main caption.
% The latest version of endfloat.sty and its documentation can obtained at:
% http://www.ctan.org/pkg/endfloat
%
% The IEEEtran \ifCLASSOPTIONcaptionsoff conditional can also be used
% later in the document, say, to conditionally put the References on a
% page by themselves.


% *** PDF, URL AND HYPERLINK PACKAGES ***
%
%\usepackage{url}
% url.sty was written by Donald Arseneau. It provides better support for
% handling and breaking URLs. url.sty is already installed on most LaTeX
% systems. The latest version and documentation can be obtained at:
% http://www.ctan.org/pkg/url
% Basically, \url{my_url_here}.


% *** Do not adjust lengths that control margins, column widths, etc. ***
% *** Do not use packages that alter fonts (such as pslatex).         ***
% There should be no need to do such things with IEEEtran.cls V1.6 and later.
% (Unless specifically asked to do so by the journal or conference you plan
% to submit to, of course. )


% correct bad hyphenation here
%\hyphenation{op-tical net-works semi-conduc-tor}


\begin{document}
	%
	% paper title
	% Titles are generally capitalized except for words such as a, an, and, as,
	% at, but, by, for, in, nor, of, on, or, the, to and up, which are usually
	% not capitalized unless they are the first or last word of the title.
	% Linebreaks \\ can be used within to get better formatting as desired.
	% Do not put math or special symbols in the title.
	\title{Introduction to Convolutional Neural Network}
	%
	%
	% author names and IEEE memberships
	% note positions of commas and nonbreaking spaces ( ~ ) LaTeX will not break
	% a structure at a ~ so this keeps an author's name from being broken across
	% two lines.
	% use \thanks{} to gain access to the first footnote area
	% a separate \thanks must be used for each paragraph as LaTeX2e's \thanks
	% was not built to handle multiple paragraphs
	%
	
	\author{Huimin Ye, 11531016
		\thanks{Huimin Ye is a PhD.~candidate with the College of Information Science and Electronic Engineering,~Zhejiang University,~Hangzhou,~China.~e-mail: 1042430841@qq.com.}
		
		% <-this % stops a space
		% <-this % stops a space
		\thanks{This report is updated on \today.}}
	
	% note the % following the last \IEEEmembership and also \thanks -
	% these prevent an unwanted space from occurring between the last author name
	% and the end of the author line. i.e., if you had this:
	%
	% \author{....lastname \thanks{...} \thanks{...} }
	%                     ^------------^------------^----Do not want these spaces!
	%
	% a space would be appended to the last name and could cause every name on that
	% line to be shifted left slightly. This is one of those "LaTeX things". For
	% instance, "\textbf{A} \textbf{B}" will typeset as "A B" not "AB". To get
	% "AB" then you have to do: "\textbf{A}\textbf{B}"
	% \thanks is no different in this regard, so shield the last } of each \thanks
	% that ends a line with a % and do not let a space in before the next \thanks.
	% Spaces after \IEEEmembership other than the last one are OK (and needed) as
	% you are supposed to have spaces between the names. For what it is worth,
	% this is a minor point as most people would not even notice if the said evil
	% space somehow managed to creep in.
	
	
	
	% The paper headers
	\markboth{Report of optimization theory using \LaTeX\ Class Files }
	{	}
	% The only time the second header will appear is for the odd numbered pages
	% after the title page when using the twoside option.
	%
	% *** Note that you probably will NOT want to include the author's ***
	% *** name in the headers of peer review papers.                   ***
	% You can use \ifCLASSOPTIONpeerreview for conditional compilation here if
	% you desire.
	
	
	
	
	% If you want to put a publisher's ID mark on the page you can do it like
	% this:
	%\IEEEpubid{0000--0000/00\$00.00~\copyright~2015 IEEE}
	% Remember, if you use this you must call \IEEEpubidadjcol in the second
	% column for its text to clear the IEEEpubid mark.
	
	
	
	% use for special paper notices
	%\IEEEspecialpapernotice{(Invited Paper)}
	
	
	
	
	% make the title area
	\maketitle
	
	% As a general rule, do not put math, special symbols or citations
	% in the abstract or keywords.
	\begin{abstract}
		This paper introduces the architecture and inference of convolutional neural network (CNN).~Commencing with a description of its multi-layer structure, followed by the derivation of feedforward computation.~For the convenience of employing gradient descent method to such an unconstrained minimization, backpropagation which is an efficient algorithm is inferred in detail.
	\end{abstract}
	
	% Note that keywords are not normally used for peerreview papers.
	\begin{IEEEkeywords}
		inference, convolution, feedforward, backpropagation, gradient descent.
	\end{IEEEkeywords}
	
	
	% For peer review papers, you can put extra information on the cover
	% page as needed:
	% \ifCLASSOPTIONpeerreview
	% \begin{center} \bfseries EDICS Category: 3-BBND \end{center}
	% \fi
	%
	% For peerreview papers, this IEEEtran command inserts a page break and
	% creates the second title. It will be ignored for other modes.
	\IEEEpeerreviewmaketitle
	
	\section{Introduction}
Supervised learning is one of the most powerful tools of Artificial Intelligence, and has led to automatic zip code recognition, speech identification, self-driving cars, and continually improving comprehension of the human genome.~Despite its significant successes, supervised learning present is still severely limited.~More specifically, most applications of it require that we manually specify the input features $x$ given to the algorithm.~Once a good feature representation is obtained, a supervised learning algorithm can work well.~But in such domains as computer vision, audio processing, and natural language processing, there are hundreds or perhaps thousands of researches who have spent years of their efforts slowly and laboriously in hand-engineering vision, audio or text features.~While much of this feature-engineering work is extremely clever, one may wonder if we can do better.

Ideally we would like to have algorithms that can automatically learn even better feature representations than the hand-engineered ones.~That is the reason why this paper describes the \textbf{sparse autoencoder} learning algorithm, which is one approach to learn features from unlabeled data autonomously.~In some domains, such as computer vision, this approach is not competitive by itself with the best hand-engineering features, but the features it can learn do turn out to be practical for a range of situations.
	
	\section{Existing Work}
A typical CNN known as LeNet was suggested by LeCun et al. \cite{lecun1995comparison}\cite{lecun1998gradient} in 1995, which had a great performance on identification of handwritten digits so that it was used by banks to process cheques, and by post offices to recognize addresses.~Ciresan et al.\cite{ciresan2012multi} exceeded previous results and first achieved near-human performance on recognition digits, in 2012.~During the same period, Krizhevsky and Hinton \cite{krizhevsky2012imagenet} designed a deep convolutional neural network and acquired the top 1 in ImageNet Large Scale Visual Recognition Challenge (ILSVRC) 2012, which revealed the power of CNN to other participants.~Moreover, a special CNN referred as DeepID \cite{Sun2014CVPR} was employed to face verification and obtained the state-of-the-art accuracy on LFW dataset.~The tasks of object detection and semantic segmentation also gained considerable improvement through the proposals of R-CNN \cite{girshick2014rich} and Fast R-CNN \cite{girshick2015fast} sequentially, in 2014 and 2015 respectively.
			
	\section{Architecture}
Typically convolutional neural network (see Fig.~\ref{fig:CNN}) is composed of pairs of convolution layer and pooling layer with a fully connected layer at the last stage of the architecture.~The input of network is a two-dimensional image and its feedforward process contains convolution and pooling (also subsampling) operations that generate the corresponding layers' feature maps.
\begin{figure}[htbp]
	\centering
	\includegraphics[width=0.5\textwidth]{figures/CNN.png}
	\caption{A convolutional neural network architecture} \label{fig:CNN}
\end{figure}
After hierarchical feedforward steps, the feature maps before the fully connected layer which have smaller size than primitive input are all concatenated into a long vector.~Considering the classification task, the output of network possesses $k$ units (the class number for discrimination) that are organized as a ``one-of-$k$'' code where only one element of output is positive and the rest of them are either zero or negative depending on the choice of output activation function.~Like other artificial neural networks, CNN has numerous adjustable parameters, namely weights and biases (notation $\bm{W}$ and $\bm{b}$), respectively.~Error backpropagation algorithm which is efficient in multi-layer structure is described below to interpret the approach of updating the parameters concretely.
	
	\section{Feedforward Pass}
\subsection{Convolution Layer}
At a convolution layer $\ell$, the previous layer's feature maps (or original input) $\bm{x}_i^{\ell-1}$ are convolved with adjustable kernels (also weights) $\bm{k}_{ij}^\ell$ and put through the activation function to form the output feature map.~In general, we have that
\begin{align}
\bm{x}_j^\ell = f \left( \sum_{i \in M_j} \bm{x}_i^{\ell-1} \ast \bm{k}_{ij}^\ell + b_j^\ell \right),
\end{align}
where $M_j$ represents a selection of input maps, e.g. pairs or triplets of combinations.~Each output map is given an additive bias $b$, but for a particular output map, the input maps will be computed with distinct kernels, see Fig.~\ref{fig:convolution}.
\begin{figure}[htbp]
	\centering
	\includegraphics[width=0.25\textwidth]{figures/convolve_kernel.png}
	\caption{Convolution operation} \label{fig:convolution}
\end{figure}
The activation function $f(\cdot)$ is commonly chosen to be the sigmoid function $f(x)=a/(1+e^{-bx})$ or the hyperbolic tangent function $f(x)=a\tanh(bx)$.~In addition, it is convenient that the derivatives of both may be represented by themselves, i.e., 
\begin{align}
	\text{sigmoid} \quad & f'(x) = f(x)(1 - f(x)), \\
	\text{tanh} \quad & f'(x) = 1 - f^2 (x).
\end{align}

\subsection{Pooling Layer}
A pooling layer produces a downsampled version of the input maps.~The subsampling operation does not alter the number of feature maps but the sizes of them.~That is, $N$ input maps will obtain exactly $N$ output maps with a smaller dimension.~More formally,
\begin{align}
	\bm{x}_j^\ell = f(\text{down}(\bm{x}_j^{\ell-1}) + b_j^\ell),
\end{align}
where $\text{down}(\cdot)$ is a pooling (or subsampling) operation.~Specifically this function will extract the average (mean pooling) or the maximum (max pooling) over each distinct $n$-by-$n$ block in an input map so that the corresponding output map is $n$-times smaller along both spatial dimensions but preserves the characteristics simultaneously.~Each output map also plus an additional bias $b$.~Fig.~\ref{fig:pooling} illustrates this procedure.
\begin{figure}[htbp]
	\centering
	\includegraphics[width=0.45\textwidth]{figures/pooling.png}
	\caption{Pooling operation} \label{fig:pooling}
\end{figure}

In the derivation that follows, we will consider the squared-error loss function.~For a multi-class problem with $c$ class and $N$ training examples, this error is given by
\begin{align}
	E^N = \frac{1}{2} \sum_{n=1}^{N} \sum_{k=1}^{c} (t_k^n - y_k^n)^2.
\end{align}
Here $t_k^n$ is the $k$-th dimension of the $n$-th pattern's corresponding label, and $y_k^n$ is similarly the value of the $k$-th output layer unit in response to the $n$-th input pattern.~Because this error over the entire dataset is just a sum over the individual errors on each pattern, we will consider it with respect to a single pattern, i.e. the $n$-th one,
\begin{align}
	E^n = \frac{1}{2} \sum_{k=1}^{c} (t_k^n - y_k^n)^2 = \frac{1}{2} \parallel \bm{t}^n - \bm{y}^n \parallel _2^2.
\end{align}
Then we formulate our problem to the standard form as follows, 
\begin{align}
\text{minimize} \quad \sum_n \frac{1}{2} \parallel \bm{t}^n - \bm{y}^n \parallel _2^ 2.
\end{align}
Note that there is no constraint though $\bm{y}^n$, which is the output vector of last layer, can be represented as a complicated function with respect to variables $W,b$.~Because of the hierarchical convolution and pooling, we are not able to understand the relationship between $\bm{y}^n$ and $W,b$, directly.~In convolutional neural network, tiny changes of $W$ somehow may lead to obvious promotion or deterioration on its performance, but gigantic variations of $W$ may not work in the same way.~In brief, it is difficult to comprehend or prove whether the problem proposed above is convex or not.

Though the convexity of our problem remains unknown (i.e., we can not find the global optimal solutions $W,b$ straightforwardly), it is still possible to employ the gradient descent method, one of the effective approaches for unconstrained minimization, to solve the problem. 

With a variety of $W$ and $b$, we do not insure that $W$ and $b$ converge eventually to their global optimal solutions instead of local optimal solutions, respectively.~In practice, it is quite probable to get to the latter case.

	
	\section{Backpropagation Pass}
In short, we update the parameters $\bm{W}$ and $\bm{b}$ in the light of rules given by
\begin{align}
	\bm{W}_{\tau+1} &= \bm{W}_\tau - \eta \frac{\partial E}{\partial \bm{W}_\tau}, \\
	\bm{b}_{\tau+1} &= \bm{b}_\tau - \eta \frac{\partial E}{\partial \bm{b}_\tau}.
\end{align}
Here $\tau$ denotes the number of iterations and learning rate (notation $\eta$) is usually a small value.~Since multi-layer architecture results in the expressions of partial derivatives with respect to $\bm{W},\bm{b}$, in former layer, being complex considerably through the chain rule, we use error backpropagation algorithm to simplify them.

The ``error'' which propagates backwards through the network can be thought of as ``sensitivities'' with respect to the current total input of each unit, yields
\begin{align}
	\bm{\delta}^\ell = \frac{\partial E}{\partial \bm{u}^\ell}, ~~ \bm{u}^\ell = \bm{W}^\ell \bm{x}^{\ell - 1} + \bm{b}^\ell, ~~ \bm{x}^\ell = f(\bm{u}^\ell).
\end{align}

\subsection{Fully connected Layer}
We obtain the error and corresponding gradients of the last layer without much effort,
\begin{align}
	\bm{\delta}^L &= (\bm{y}^n - \bm{t}^n) \circ f'(\bm{u}^L), \label{eq:delta_L}\\
	\frac{\partial E}{\partial \bm{W}^L} &= \bm{\delta}^L (\bm{x}^{L-1})^T, ~~ \frac{\partial E}{\partial \bm{b}^L} = \bm{\delta}^L.
\end{align}
In \eqref{eq:delta_L}, ``$\circ$'' denotes the Hadamard product (i.e. element-wise multiplication).~From that, the partial derivative with respect to $\bm{W}$ is computed as an outer product between the vector of input and sensitivities, and more compact for $\bm{b}$.~Then this error is passed to previous layer by means of the recurrence relation below, 
\begin{align}
	\bm{\delta}^{L-1} = (\bm{W}^L)^T \bm{\delta}^L \circ f'(\bm{u}^{L-1}). \label{eq:recursion}
\end{align}

\subsection{Pooling Layer}
Having acquired the sensitivities for each map of the pooling layer $\ell$, we immediately compute the gradient for $\bm{b}$ by summing over all elements simply in a sensitivities map,
\begin{align}
	\frac{\partial E}{\partial b_j} = \sum_{u,v} (\bm{\delta}_j^\ell)_{u,v}.
\end{align}
Assuming that it is a convolution layer $\ell-1$ that is distributed before a pooling layer $\ell$.~Hence a block of pixels in convolution layer's map is associated with one pixel in present layer's map.~In case of mean pooling operation, to propagate sensitivities back efficiently, we extend the sensitivity map's dimension (and average its value) in layer $\ell$ to make it exactly same size as prior layer $\ell-1$, then multiply this quantity element-wise by the derivative of activation function evaluated at $\ell-1$ layer's pre-activation input, $\bm{u}$.~We can repeat the same computation for each map $j$ between two layers, see Fig.~\ref{fig:delta_pooling} for details.
\begin{align}
	\bm{\delta}_j^{\ell-1} &= up(\bm{\delta}_j^\ell) \circ f'(\bm{u}_j^{\ell-1}), \\
	up(\bm{x}) &= \bm{x} \otimes \bm{1}_{n \times n} / n^2.
\end{align}
Note that ``$\otimes$'' represents the Kronecker product, and $up(\cdot)$ denotes an upsampling operation that duplicates each pixel in the input horizontally and vertically $n$ times to the output (i.e. to be a $n \times n$ matrix) when the factor of pooling is $n$.
\begin{figure}[htbp]
	\centering
	\includegraphics[width=0.4\textwidth]{figures/backprop_2.png}
	\caption{Backpropagation in pooling layer} \label{fig:delta_pooling}
\end{figure}

\subsection{Convolution Layer} 
In case of obtained the sensitivities for each map of the convolution layer $\ell$, we compute the gradients for kernel weights $\bm{k}_{mn}^\ell$ and $\bm{b}$ using backpropagation where weights are shared across many connections.~Thus, we sum the gradients, for a given weight $\bm{k}_{mn}^\ell$, over all connections that are related to this weight.~We do not repeat interpreting the gradients for $\bm{b}$ due to the same method of calculation showed above.
\begin{align}
	\frac{\partial E}{\partial \bm{k}_{mn}^\ell} = \sum_{u,v} (\bm{\delta}_j^\ell)_{uv} (\bm{p}_{mn}^{\ell-1})_{uv},~\frac{\partial E}{\partial b_j} = \sum_{u,v} (\bm{\delta}_j^\ell)_{u,v}. \label{eq:kernel_mn}
\end{align}
In \eqref{eq:kernel_mn}, the \emph{patch} denoted by $\bm{p}_{mn}^{\ell-1}$ is a part matrix of $\bm{x_i}^{\ell-1}$ that was multiplied element-wise by $\bm{k}_{mn}^\ell$ during convolution in feedforward pass in order to compute the corresponding units in the output map $\bm{x}_j^{\ell}$.~Fig.~\ref{fig:kernel_mn} demonstrates this process concretely.
\begin{figure}[htbp]
	\centering
	\includegraphics[width=0.4\textwidth]{figures/backprop_delta_2.png}
	\caption{Backpropagation in convolution layer} \label{fig:kernel_mn}
\end{figure}

Supposing that a pooling layer $\ell-1$ is followed by a convolution layer $\ell$.~The difficulty lies in computing the sensitivity maps that propagates back to the previous layer $\ell-1$.~Here, we figure out how a given unit in the previous layer's sensitivity map $\bm{\delta}_i^{\ell-1}$ participates in generating the patch in the current layer's sensitivity map $\bm{\delta}_j^\ell$ through kernels mentioned.~Therefore, we apply a recursion rule that looks something like equation \eqref{eq:recursion}.~Observe that the weights multiplying the connections between two layers are exactly the rotated convolution kernels.~The operation goes below
\begin{align}
	\bm{\delta}_i^{\ell-1} = (\text{rot180}(\bm{K}_{ij}^\ell) \circledast \bm{\delta}_j^\ell) \circ f'(\bm{u}_i^{\ell-1}),
\end{align}
where ``$\circledast$'' denotes a full convolution operation that will automatically pad the missing inputs with zeros.~Notice that the kernel $\bm{K}_{ij}^\ell$ is rotated before to make the convolution perform cross-correlation actually.~Fig.~\ref{fig:delta_convolution} yields the detailed procedure.
\begin{figure}[htbp]
	\centering
	\includegraphics[width=0.4\textwidth]{figures/backprop_delta_1.png}
	\caption{Sensitivities backpropagation process} \label{fig:delta_convolution}
\end{figure}

			
	\section{Conclusion}
After analyzing sequential steps of feedforward and backpropagation separately, we conclude that the training approach of convolutional neural network through gradient descent method is summarized as follows
\begin{table}[htbp]
\centering
\begin{tabular}{l}
	\hline
	Training procedure of \emph{Convolutional Neural Network} \\
	\hline
	Initialize weights $\bm{W}$ and biases $\bm{b}$ in each layer with stochastic values. \\
	\textbf{Repeat} \\
	\quad 1.~Compute activation forwards through every element in each layer. \\
	\quad 2.~Evaluate Loss function $E$ of the last layer's output. \\
	\quad 3.~Compute backwards the gradients for weights $\bm{W}$ and biases $\bm{b}$ in each layer, respectively. \\
	\quad 4.~Update parameters $\bm{W}$ and $\bm{b}$ through gradient descent method. \\
	\textbf{Until} \\
	\quad Loss function $E$ is smaller than threshold pre-determined, or the maximum number of iterations reaches. \\
	\hline
\end{tabular}
%\caption{} \label{}
\end{table}

	
	% An example of a floating figure using the graphicx package.
	% Note that \label must occur AFTER (or within) \caption.
	% For figures, \caption should occur after the \includegraphics.
	% Note that IEEEtran v1.7 and later has special internal code that
	% is designed to preserve the operation of \label within \caption
	% even when the captionsoff option is in effect. However, because
	% of issues like this, it may be the safest practice to put all your
	% \label just after \caption rather than within \caption{}.
	%
	% Reminder: the "draftcls" or "draftclsnofoot", not "draft", class
	% option should be used if it is desired that the figures are to be
	% displayed while in draft mode.
	%
	%\begin{figure}[!t]
	%\centering
	%\includegraphics[width=2.5in]{myfigure}
	% where an .eps filename suffix will be assumed under latex,
	% and a .pdf suffix will be assumed for pdflatex; or what has been declared
	% via \DeclareGraphicsExtensions.
	%\caption{Simulation results for the network.}
	%\label{fig_sim}
	%\end{figure}
	
	% Note that the IEEE typically puts floats only at the top, even when this
	% results in a large percentage of a column being occupied by floats.
	
	
	% An example of a double column floating figure using two subfigures.
	% (The subfig.sty package must be loaded for this to work.)
	% The subfigure \label commands are set within each subfloat command,
	% and the \label for the overall figure must come after \caption.
	% \hfil is used as a separator to get equal spacing.
	% Watch out that the combined width of all the subfigures on a
	% line do not exceed the text width or a line break will occur.
	%
	%\begin{figure*}[!t]
	%\centering
	%\subfloat[Case I]{\includegraphics[width=2.5in]{box}%
	%\label{fig_first_case}}
	%\hfil
	%\subfloat[Case II]{\includegraphics[width=2.5in]{box}%
	%\label{fig_second_case}}
	%\caption{Simulation results for the network.}
	%\label{fig_sim}
	%\end{figure*}
	%
	% Note that often IEEE papers with subfigures do not employ subfigure
	% captions (using the optional argument to \subfloat[]), but instead will
	% reference/describe all of them (a), (b), etc., within the main caption.
	% Be aware that for subfig.sty to generate the (a), (b), etc., subfigure
	% labels, the optional argument to \subfloat must be present. If a
	% subcaption is not desired, just leave its contents blank,
	% e.g., \subfloat[].
	
	
	% An example of a floating table. Note that, for IEEE style tables, the
	% \caption command should come BEFORE the table and, given that table
	% captions serve much like titles, are usually capitalized except for words
	% such as a, an, and, as, at, but, by, for, in, nor, of, on, or, the, to
	% and up, which are usually not capitalized unless they are the first or
	% last word of the caption. Table text will default to \footnotesize as
	% the IEEE normally uses this smaller font for tables.
	% The \label must come after \caption as always.
	%
	%\begin{table}[!t]
	%% increase table row spacing, adjust to taste
	%\renewcommand{\arraystretch}{1.3}
	% if using array.sty, it might be a good idea to tweak the value of
	% \extrarowheight as needed to properly center the text within the cells
	%\caption{An Example of a Table}
	%\label{table_example}
	%\centering
	%% Some packages, such as MDW tools, offer better commands for making tables
	%% than the plain LaTeX2e tabular which is used here.
	%\begin{tabular}{|c||c|}
	%\hline
	%One & Two\\
	%\hline
	%Three & Four\\
	%\hline
	%\end{tabular}
	%\end{table}
	
	
	% Note that the IEEE does not put floats in the very first column
	% - or typically anywhere on the first page for that matter. Also,
	% in-text middle ("here") positioning is typically not used, but it
	% is allowed and encouraged for Computer Society conferences (but
	% not Computer Society journals). Most IEEE journals/conferences use
	% top floats exclusively.
	% Note that, LaTeX2e, unlike IEEE journals/conferences, places
	% footnotes above bottom floats. This can be corrected via the
	% \fnbelowfloat command of the stfloats package.
	
	
	% if have a single appendix:
	%\appendix[Proof of the Zonklar Equations]
	% or
	%\appendix  % for no appendix heading
	% do not use \section anymore after \appendix, only \section*
	% is possibly needed
	
	% use appendices with more than one appendix
	% then use \section to start each appendix
	% you must declare a \section before using any
	% \subsection or using \label (\appendices by itself
	% starts a section numbered zero.)
	%
	
	
	%\appendices
	%\section{Proof of the First Zonklar Equation}
	%Appendix one text goes here.
	%
	%% you can choose not to have a title for an appendix
	%% if you want by leaving the argument blank
	%\section{}
	%Appendix two text goes here.
	%
	%
	%% use section* for acknowledgment
	%\section*{Acknowledgment}
	%
	%
	%The authors would like to thank...
	
	
	% Can use something like this to put references on a page
	% by themselves when using endfloat and the captionsoff option.
	\ifCLASSOPTIONcaptionsoff
	\newpage
	\fi
	
	
	
	% trigger a \newpage just before the given reference
	% number - used to balance the columns on the last page
	% adjust value as needed - may need to be readjusted if
	% the document is modified later
	%\IEEEtriggeratref{8}
	% The "triggered" command can be changed if desired:
	%\IEEEtriggercmd{\enlargethispage{-5in}}
	
	% references section
	
	% can use a bibliography generated by BibTeX as a .bbl file
	% BibTeX documentation can be easily obtained at:
	% http://mirror.ctan.org/biblio/bibtex/contrib/doc/
	% The IEEEtran BibTeX style support page is at:
	% http://www.michaelshell.org/tex/ieeetran/bibtex/
	\bibliographystyle{IEEEtran}
	% argument is your BibTeX string definitions and bibliography database(s)
	\bibliography{mybibliography}
	%
	% <OR> manually copy in the resultant .bbl file
	% set second argument of \begin to the number of references
	% (used to reserve space for the reference number labels box)
	
	%\begin{thebibliography}{1}
	%
	%\bibitem{IEEEhowto:kopka}
	%H.~Kopka and P.~W. Daly, \emph{A Guide to \LaTeX}, 3rd~ed.\hskip 1em plus
	%  0.5em minus 0.4em\relax Harlow, England: Addison-Wesley, 1999.
	%
	%\end{thebibliography}
	
	% biography section
	%
	% If you have an EPS/PDF photo (graphicx package needed) extra braces are
	% needed around the contents of the optional argument to biography to prevent
	% the LaTeX parser from getting confused when it sees the complicated
	% \includegraphics command within an optional argument. (You could create
	% your own custom macro containing the \includegraphics command to make things
	% simpler here.)
	%\begin{IEEEbiography}[{\includegraphics[width=1in,height=1.25in,clip,keepaspectratio]{mshell}}]{Michael Shell}
	% or if you just want to reserve a space for a photo:
	
	%\begin{IEEEbiography}{Michael Shell}
	%Biography text here.
	%\end{IEEEbiography}
	%
	%% if you will not have a photo at all:
	%\begin{IEEEbiographynophoto}{John Doe}
	%Biography text here.
	%\end{IEEEbiographynophoto}
	
	% insert where needed to balance the two columns on the last page with
	% biographies
	%\newpage
	
	%\begin{IEEEbiographynophoto}{Jane Doe}
	%Biography text here.
	%\end{IEEEbiographynophoto}
	
	% You can push biographies down or up by placing
	% a \vfill before or after them. The appropriate
	% use of \vfill depends on what kind of text is
	% on the last page and whether or not the columns
	% are being equalized.
	
	%\vfill
	
	% Can be used to pull up biographies so that the bottom of the last one
	% is flush with the other column.
	%\enlargethispage{-5in}
	
	% that's all folks
\end{document}
