\section{Existing Work}
A typical CNN known as LeNet was suggested by LeCun et al. \cite{lecun1995comparison}\cite{lecun1998gradient} in 1995, which had a great performance on identification of handwritten digits so that it was used by banks to process cheques, and by post offices to recognize addresses.~Ciresan et al.\cite{ciresan2012multi} exceeded previous results and first achieved near-human performance on recognition digits, in 2012.~During the same period, Krizhevsky and Hinton \cite{krizhevsky2012imagenet} designed a deep convolutional neural network and acquired the top 1 in ImageNet Large Scale Visual Recognition Challenge (ILSVRC) 2012, which revealed the power of CNN to other participants.~Moreover, a special CNN referred as DeepID \cite{Sun2014CVPR} was employed to face verification and obtained the state-of-the-art accuracy on LFW dataset.~The tasks of object detection and semantic segmentation also gained considerable improvement through the proposals of R-CNN \cite{girshick2014rich} and Fast R-CNN \cite{girshick2015fast} sequentially, in 2014 and 2015 respectively.